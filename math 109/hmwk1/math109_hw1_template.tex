\documentclass{homework}

\newcommand{\AND}{\text{ and }}
\newcommand{\OR}{\text{ or }}
\newcommand{\NOT}{\text{not }}
\newcommand{\divides}{\text{ divides }}

\name{Your Name}
\course{Math 109}
\term{Summer Academy 2016}
\hwnum{1}

\begin{document}

\begin{problem}[Eccles I.4, p.53]
Prove the following statements concerning positive integers $a$, $b$, and $c$.
\begin{parts}[r]
\part
\label{I.4.i}
$(a \divides b) \AND (a \divides c) \implies a \divides (b + c)$.
\part
\label{I.4.ii}
$(a \divides b) \OR (b \divides c) \implies a \divides b c$.
\end{parts}
\end{problem}

\begin{solution}
\ref{I.4.i}

\ref{I.4.ii}

\end{solution}

\begin{problem}[Eccles I.6, p. 54]
Use the properties of addition and multiplication of real numbers given in
Properties 2.3.1 to deduce that, for all real numbers $a$ and $b$,
\begin{parts}[r]
\part
\label{I.6.a}
$a \times 0 = 0 = 0 \times a$,
\part
\label{I.6.b}
$(-a) b = -a b = a (-b)$,
\part
\label{I.6.c}
$(-a) (-b) = a b$.
\end{parts}
\end{problem}

\begin{solution}
\ref{I.6.a}

\ref{I.6.b}

\ref{I.6.c}

\end{solution}

\begin{problem}[Eccles I.10, p. 54]
What is wrong with the following proof that $1$ is the largest integer?
\begin{quote}
Let $n$ be the largest integer.
Then, since $1$ is an integer we must have $1 \leq n$.
On the other hand, since $n^2$ is also an integer we must have $n^2 \leq n$ from
which it follows that $n \leq 1$.
Thus, since $1 \leq n$ and $n \leq 1$ we must have $n = 1$.
Thus $1$ is the largest integer as claimed.
\end{quote}
What does this argument prove?
\end{problem}

\begin{solution}

\end{solution}

\begin{problem}[Eccles I.19, p. 56]
Prove that
\begin{equation*}
\prod_{i=2}^n \left(1 - \frac{1}{i^2}\right) = \frac{n+1}{2 n}
\end{equation*}
for integers $n \geq 2$.
\end{problem}

\begin{solution}

\end{solution}

\begin{problem}[Eccles I.21, p. 56]
Suppose that $x$ is a real number such that $x + 1/x$ is an integer.
Prove by induction on $n$ that $x^n + 1/x^n$ is an integer for all positive
integers $n$.
[For the inductive step consider $(x^k + 1/x^k)(x + 1/x)$.]
\end{problem}

\begin{solution}

\end{solution}

\begin{problem}[Eccles I.25, p. 57]
Let $u_n$ be the $n$th Fibonacci number (Definition 5.4.2).
Prove, by \emph{induction on $n$} (without using the Binet formula Proposition
5.4.3), that
\begin{equation*}
u_{m + n} = u_{m-1} u_n + u_m u_{n+1}
\end{equation*}
for all positive integers $m$ and $n$.

Deduce, again using induction on $n$, that $u_m$ divides $u_{m n}$.
\end{problem}

\begin{solution}

\end{solution}

\end{document}
